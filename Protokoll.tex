\documentclass[11pt,ngerman,a4paper]{article}
%Gummi|061|=)
\usepackage{amsmath}
\usepackage{a4wide}
\usepackage{amsthm}
\usepackage{amsbsy}
\usepackage{amssymb}
\usepackage{url}
\usepackage{inputenc}
\usepackage{graphicx}
\usepackage{selinput}
\usepackage{here}
\usepackage{paralist}
\usepackage{rotating} 
\SelectInputMappings{
adieresis={ä},
germandbls={ß},
}
\title{\textbf{Versuch V201: Das Dulong-Petitsche Gesetz}}
\author{Martin Bieker\\
		Julian Surmann\\
		\\
		Durchgef\"{u}hrt am 21.11.2013\\
		TU Dortmund}
\date{}
\usepackage{graphicx}
\begin{document}
\renewcommand\tablename{Tabelle}
\renewcommand\figurename{Abbildung}
\maketitle
\thispagestyle{empty}
\newpage
\clearpage
\setcounter{page}{1}


\section{Einleitung}

Im folgenden Versuch wird die Molw\"arme verschiedener Festk\"orper mit einem Kalorimeter bestimmt werden.  Diese Ergebnisse kl\"aren ob die Bewegung der Atome in Festk\"orperen mit Hilfe der Quantenmechanik beschreiben werden muss oder ob ein klassisches Modell ausreichend ist.
\section{Theorie}

Die spezifische W\"armekapazit\"at gibt an, welche W\"armemenge $\Delta Q$ ein K\"orper bei einer Temoeratur\"anderung aufnimmt oder abgibt. Es gilt:
\begin{equation}
c = \frac{\Delta Q}{m \cdot\Delta T}
\end{equation}
Wird diese Gr\"o\ss e auf ein Mol des Stoffes bezogen, erh\"alt man die so genante Mol\"arme C mit 

\begin{equation}
C = \frac{\Delta Q }{n\cdot \Delta T} =\frac{M\cdot \Delta Q}{m\cdot \Delta T} = M  \cdot c.
\end{equation}
\subsection{}
\section{Versuchsdurchf\"uhrung}

\subsection{Aufbau}
Zur Messung der W\"armekapazitaeten wird ein Kalorimeter verwendet. 
Bei dieser Apparatur handelt es sich um ein gut isoliertes Gef\"a\ss\ in dem sich eine genau bestimmte Menge Wasser der Temperatur $T_{Kalt}$ befindet. Der zu untersuchende K\"orper wird in einem Wasserbad auf eine Temperatur $T_{Heiss}$ von circa $100 ^\circ C$ erw\"armt. Danach wird dieser in das Wasser des Kalorimeters eingedaucht. Im Folgenden wid das Gef\"a\ss\ mit einem Deckel verschlossen. Die Temperatur des Systems wird mit dem Thermoelment gemssen
\section{Auswertung}

\section{Diskussion}


\section{Literaturverzeichnis}

\section{Anhang}

\end{document}
