\documentclass[11pt,ngerman,a4paper]{article}
%Gummi|061|=)
\usepackage{amsmath}
\usepackage{a4wide}
\usepackage{amsthm}
\usepackage{amsbsy}
\usepackage{amssymb}
\usepackage{url}
\usepackage{inputenc}
\usepackage{graphicx}
\usepackage{selinput}
\usepackage{float}
\SelectInputMappings{
adieresis={ä},
germandbls={ß},
}
\title{\textbf{Versuch V201: Das Dulong-Petitsche Gesetz}}
\author{Martin Bieker\\
		Julian Surmann\\
		\\
		Durchgef\"{u}hrt am 21.11.2013\\
		TU Dortmund}
\date{}
\usepackage{graphicx}
\begin{document}
\renewcommand\tablename{Tabelle}
\renewcommand\figurename{Abbildung}
\maketitle
\thispagestyle{empty}
\newpage
\clearpage
\setcounter{page}{1}


\section{Einleitung}

Im folgenden Versuch wird die Molw\"arme verschiedener Festk\"orper mit einem Kalorimeter bestimmt werden.  Diese Ergebnisse kl\"aren ob die Bewegung der Atome in Festk\"orperen mit Hilfe der Quantenmechanik beschreiben werden muss oder ob ein klassisches Modell ausreichend ist.
\section{Theorie}

Die spezifische W\"armekapazit\"at gibt an, welche W\"armemenge $\Delta Q$ ein K\"orper bei einer Temoeratur\"anderung aufnimmt oder abgibt. Es gilt:
\begin{equation}
c = \frac{\Delta Q}{m \cdot\Delta T}
\end{equation}
Wird diese Gr\"o\ss e auf ein Mol des Stoffes bezogen, erh\"alt man die so genante Mol\"arme C mit 

\begin{equation}
C = \frac{\Delta Q }{n\cdot \Delta T} =\frac{M\cdot \Delta Q}{m\cdot \Delta T} = M  \cdot c.
\end{equation}
\subsection{}
\section{Versuchsdurchf\"uhrung}

\subsection{Aufbau}
Zur Messung der W\"armekapazitaeten wird ein Kalorimeter verwendet. 
Bei dieser Apparatur handelt es sich um ein gut isoliertes Gef\"a\ss\ in dem sich eine genau bestimmte Menge Wasser der Temperatur $T_{Kalt}$ befindet. Der zu untersuchende K\"orper wird in einem Wasserbad auf eine Temperatur $T_{Heiss}$ von circa $100 ^\circ C$ erw\"armt. Danach wird dieser in das Wasser des Kalorimeters eingedaucht. Im Folgenden wid das Gef\"a\ss\ mit einem Deckel verschlossen. Die Temperatur des Systems wird mit dem Thermoelment gemssen
\section{Auswertung}
\subsection{Wärmekapazität des Kalorimeters}
Um im zweiten Versuchsteil korrekte Wärmekapazitäten für die Metalle zu errechnen, wurde im ersten Versuchsteil die Wärmekapazität $c_gm_g$ des Kalorimeters bestimmt. Dazu wurden folgende Messwerte ermittelt:
\begin{table}[H]
\centering
\begin{tabular}{|c|c|c|c|c|c|c|c|}
\hline
$U_{kalt}[mV]$ &$U_{warm}[mV]$ &$U_{misch}[mV]$ & $T_{kalt}[K]$ & $T_{warm}[K]$ & $T_{misch}[K]$ & $m_{kalt}[Kg]$ & $ m_{warm} [Kg]$ \\
\hline
0.85 & 4.10 & 2.06 & 294.40 & 373.10 & 324.17 & 0.37292 & 0.27308\\
0.91 & 4.10 & 2.13 & 295.89 & 373.10 & 325.87 & 0.37441 & 0.27173\\
0.90 & 4.09 & 2.12 & 295.64 & 372.86 & 325.63 & 0.37300 & 0.27305\\
\hline
\end{tabular}
\caption{Wärmekapazität des Kalorimeters - Messwerte}
\end{table}
\noindent
Zunächst ist es erforderlich, die gemessenen Spannungen $U_{kalt}$, $U_{warm}$ und $U_{misch}$ in Temperaturen umzurechnen. Dazu werden die Spannungen in die gegebene Interpolationsformel (Formel (XYZ)) eingesetzt. Die Wärmekapazität $c_gm_g$ des Kalorimeters wird dann durch einsetzen der Messwerte in Formel (XYZ) bestimmt:
\begin{table}[H]
\centering
\begin{tabular}{|l|c|c|c|}
\hline
$c_gm_g$ & 317.36 & 223.83 & 238.42 \\
\hline
\end{tabular}
\caption{Wärmekapazität des Kalorimeters - Ergebnisse}
\end{table}
\noindent
Mit Hilfe der Formel $ \overline{x}=\frac{1}{N}\sum_{i}^N x_i$ wird dann der Mittelwert $\overline{c_gm_g}$ ermittelt:
\[ \overline{c_gm_g}= (260 \pm 29)\,\frac{J}{K}. \]
Der Fehler des Mittelwertes wurde mit $\Delta \overline{x}=\sqrt{\frac{1}{N(N-1)}\sum_{i}^N(x_i-\overline{x})^2}$ berechnet.
\subsection{Wärmekapazität von Metallen}
Im zweiten Versuchsteil wurden je drei Messungen mit Blei und Aluminium ausgeführt. In der Tabelle (XYZ) sind alle Messergebnisse für die Spannungen sowie die berechneten Temperaturen aufgelistet.
\begin{table}[H]
\centering
\begin{tabular}{|c|c|c|c|c|c|c|}
\hline
Metall & $U_{Wasser}[mV]$ &$U_{Metall}[mV]$ &$U_{misch}[mV]$ & $T_{Wasser}[K]$ & $T_{Metall}[K]$ & $T_{misch}[K]$ \\
\hline
Blei & 0.89 & 4.1 & 0.95 & 295.39 & 373.10 & 296.88\\
Blei & 1.04 & 4.1 & 1.12 & 299.11 & 373.10 & 301.09\\
Blei & 0.83 & 4.09 & 0.91 & 293.90 & 372.68 & 295.89\\
Aluminium & 0.95 & 4.1 & 1.04 & 296.88 & 373.10 & 299.11\\
Aluminium & 1.12 & 4.09 & 1.22 & 301.09 & 372.86 & 303.56\\
Aluminium & 0.90 & 4.09 & 1.01 & 295.64 & 372.86 & 298.36\\
\hline
\end{tabular}
\caption{Wärmekapazität der Metalle - Messwerte}
\end{table}
\noindent
Darüber hinaus wurden folgende Daten ermittelt:
\begin{itemize}
\item $m_{Wasser} = 0.60815 \,Kg$
\item $m_{Blei} = 0.58765 \, Kg$
\item $m_{Aluminium} = 0.11380 \, Kg$
\end{itemize}
Aus den gemessenen Werten lassen sich jetzt mit Hilfe der Formel (XYZ) alle $c_k$ errechnen:
\begin{table}[H]
\centering
\begin{tabular}{|c|c|}
\hline
Material & $c_k [\frac{J}{Kg*K}]$ \\
\hline
Blei & $(93.1 \pm 1.0)$ \\
Blei & $(131.1\pm1.4)$ \\
Blei & $(123.1\pm1.3)$\\
Aluminium & $(742\pm8)$ \\
Aluminium & $(878\pm9)$\\
Aluminium & $(902\pm9)$\\
\hline
\end{tabular}
\caption{Wärmekapazität des Kalorimeters - Ergebnisse}
\end{table}
\noindent
Die Fehler wurden jeweils mit
\[ \Delta c_k=\frac{\partial f}{\partial c_gm_g}\Delta c_gm_g = \frac{ \left( T_{misch}-T_{Wasser} \right) }{m_{Metall} \left( T_{Metall}-T_{misch}\right)} \Delta c_gm_g\]
berechnet, dabei steht $f$ für die Formel (XYZ).
Die Bildung der Mittelwerte ergibt für
\begin{itemize}
\item Blei: $c_k = (116\pm2)\, \frac{J}{Kg*K}$
\item Aluminium: $c_k = (841\pm15)\, \frac{J}{Kg*K}$.
\end{itemize}
Der Fehler des Mittelwertes wird hier errechnet mit $\Delta \overline{c_k}=\sqrt{\sum_{i}^3 \left(\frac{\Delta c_{k_i}}{3}\right)^2}$.
\newline
Nun soll $c_k$ umgerechnet werden in die Atomwärme $C_p$ (Atomwärme bei konstantem Druck).
Mit Nutzung der Formel (XYZ) ergeben sich
\begin{table}[H]
\centering
\begin{tabular}{|c|c|c|}
\hline
Material & $M\,[\frac{Kg}{Mol}]$ & $C_p \, [\frac{J}{KMol}]$\\
\hline
Blei & $ 0.2072 $ & $(24.0\pm0.4)$\\
Aluminium & $ 0.0270 $ & $(22.7\pm0.4)$\\

\hline
\end{tabular}
\caption{Atomwärme $C_p$ bei konstantem Druck}
\end{table}
\noindent
Das eigentliche Versuchsziel ist ein Vergleich der Atomwärme bei konstantem Volumen mit dem Dulong-Petit'schen Gesetz. Um die Atomwärme bei konstantem Volumem zu erhalten, wird die Formel (XYZ) nach $C_v$ umgeformt. Für $C_v$ ergeben sich bei
\begin{itemize}
\item Blei: $C_v = (24.0\pm0.4)[J/(K Mol)]$
\item Aluminium: $C_v = (22.7\pm0.4)[J/(K Mol)]$.
\end{itemize}
Die Fehler von $C_v$ wurden mit python uncertainties ermittelt.
\newline

\section{Diskussion}


\section{Literaturverzeichnis}

\section{Anhang}

\end{document}
